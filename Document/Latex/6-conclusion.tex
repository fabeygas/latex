\chapter{สรุปและข้อเสนอแนะ}

การดำเนินโครงงานเพื่อพัฒนาแอพลิเคชันบีอิงเวลเนส นี้ พบว่าระบบสามารถทำงานได้ตามที่วิเคราะห์และออกแบบไว้ แต่ก็พบปัญหาและอุปสรรคระหว่างการพัฒนา ในบทนี้ผู้พัฒนาจึงขอสรุปความสามารถของระบบ ชี้แจงปัญหาและอุปสรรค พร้อมเสนอแนวทางในการพัฒนารแอปพลิเคชัน ต่อ ตามลำดับ

\section{สรุปความสามารถของระบบ}
\begin{enumerate}

			\item สมาชิก
			\begin{itemize}
				\item สมัครสมาชิกได้
				\item เข้าสู่ระบบได้
				\item ดูข้อมูลการบริโภคในรูปแบบวัน สัปดาห์และเดือนได้
				\item  เพิ่มอาหารที่ไม่มีในระบบได้ 
				\item  สแกนอาหารและบันทึกอาหารได้
				\item  เปลี่ยน E-mail Password และลบผู้ใช้ได้ 
				\item  ออกจากระบบได้ 
			\end{itemize}
		\end{enumerate}
	
\section{ปัญหาและอุปสรรคในการพัฒนา}
  \begin{enumerate}
   \item การระบุโรคมีความแม่นยำน้อยเนื่องจากต้องมีการนำ อายุ น้ำหนัก ส่วนสูงมาร่วมในการคำนวณ \\ 
   แนวทางการแก้ไข : เปลี่ยนมาเป็น Progrssbar ให้ระบุการบริโภคสูงสุดในแต่ละวัน 

  \end{enumerate}

\section{แนวทางการพัฒนาต่อ}
\begin{enumerate}
	\item ปรับแต่ง UI ให้น่าใช้งาน
 	\item พัฒนาฟังก์ชันระบุโรคให้มีความแม่นยำและนำมาใช้ได้จริง
 	\item พัฒนาฟังก์ชันการออกกำลังกาย
\end{enumerate}




