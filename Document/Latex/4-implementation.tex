\chapter{การพัฒนาระบบ}
หลังจากที่ได้มีการเตรียมความพร้อมสำหรับการพัฒนาในด้านต่าง ไม่ว่าจะเป็นที่มาและความสำคัญของปัญหา เทคโนโลยีที่มีความเหมาะสมกับระบบ และการออกแบบระบบการทำงานรวมไปถึงโครงสร้างของข้อมูล ในบทนี้จะเป็นการพูดถึงการสร้างระบบที่ได้มีการออกแบบไว้ในบทที่แล้วจะถูกนำเสนอในบทนี้ ซึ่งจะอธิบายถึงตัวอย่างการทำงานของแอปพลิเคชันดังนี้
		\section{โครงสร้างของการสร้างหน้า MainActivity}
		\begin{figure}[H]
			{\setstretch{1.0}\begin{lstlisting}
Scan scan;
Addfood addfood;
Dashboard dash;
Pagerdash pagerdash;
BottomNavigationView nv;
				\end{lstlisting}}
			\caption{ตัวแปรในคลาส MainActivity}
			\label{Fig:MainActivity}
		\end{figure}
		จากภาพที่ \ref{Fig:MainActivity} ตัวแปรที่ประกาศขึ้นเพื่อใช้ในการทำงานของคลาส MainActivity สามารถอธิบายได้ดังนี้
		\begin{itemize}[label={--}]
			\item บรรทัดที่ 1 ตัวแปร scan ใช้แสดงผลหน้าสแกน
			\item บรรทัดที่ 2 ตัวแปร addfood ใช้แสดงผลหน้าจอเพิ่มอาหารโดยสมาชิก  
			\item บรรทัดที่ 3 ตัวแปร dash ใส้แสดงผลหน้าจอแดชบอร์ด
			\item บรรทัดที่ 4 ตัวแปร pagerdash เป็นตัวแปลงหน้าจอเมื่อผู้ใช้เลือก Tab Week หรือ Tab Month
			\item บรรทัดที่ 5 ตัวแปร nv เป็นที่เก็บเมนู Scan และ Myfood มีไว้ให้สมาชิกเลือก
		
		\end{itemize}

		

		\begin{figure}[H]
			{\setstretch{1.0}\begin{lstlisting}
        @Override
 public boolean onNavigationItemSelected(@NonNull MenuItem menuItem) {
     switch (menuItem.getItemId()) {
         case R.id.cam:
             if (staurselect != "scan") {
                 ft = getSupportFragmentManager().beginTransaction().replace(R.id.Framefag, scan);
                 ft.commit();
                 staurselect = "scan";
             }
             break;
         case R.id.dash:
             if (staurselect != "dashbord" ) {
                 Calories = Sugar = Sodium = Fat = submit = null;
                 ft = getSupportFragmentManager().beginTransaction().replace(R.id.Framefag, new Pagerdash().newInstance(Calories,Sugar,Sodium,Fat,submit));
                 ft.commit();
                 staurselect = "dashbord";
             }
             break;
         case  R.id.cart:
             if (staurselect != "Addfood"){
                 ft = getSupportFragmentManager().beginTransaction().replace(R.id.Framefag, addfood );
                 ft.commit();
                 staurselect = "Addfood";
             }
             break;
     }
     return true;
 }

				\end{lstlisting}}
			\caption{โค้ดส่วนที่ใช้ในการสร้างเมนูนำทางหลักภายในคลาส MainActivity}
			\label{Fig:MainActivity2}
		\end{figure}
\newpage
		จากภาพที่ \ref{Fig:MainActivity2} สามารถอธิบายการทำงานโค้ดส่วนที่ใช้เมนูนำทางหลักภายในคลาส MainActivity ได้ดังนี้
		\begin{itemize}[label={--}]
			\item บรรทัดที่ 1     เป็นการเพิ่มการดักจับอีเวนต์ (Event) เพื่อสลับหน้าจอการแสดงผลที่เกิดขึ้นเมื่อผู้ใช้กดที่เมนูนำทาง
			\item บรรทัดที่ 2-3   เป็นการเรียกใช้เมธอดชันจากคลาสที่ extends มา
			\item บรรทัดที่ 4     เป็นการรับ ID 
			\item บรรทัดที่ 5    	เป็นการระบุ ID ไปยังไอคอน cam ที่ไฟล์ xml
			\item บรรทัดที่ 6-9 	เป็นการแสดงหน้าจอสแกนและป้องกันการกดหน้าสแกนซ้ำ
			\item บรรทัดที่ 10 		เป็นการหยุดการทำงานของ case เพื่อไม่ให้ทำงานต่อเนื่อง 
			\item บรรทัดที่ 11	  เป็นการระบุไปยังไอคอน dash ที่ไฟล์ xml
			\item บรรทัดที่ 12-17 เป็นการแสดงหน้าจอแดชบอร์ดและป้องกันการกดหน้าสแกนซ้ำ
			\item บรรทัดที่ 18    เป็นการหยุดการทำงานของ case เพื่อไม่ให้ทำงานต่อเนื่อง 
			\item บรรทัดที่ 19    เป็นการระบุ ID ไปยังไอคอน cart ที่ไฟล์ xml
			\item บรรทัดที่ 20-24 เป็นการแสดงหน้าจอเพิ่มอาหารและป้องกันการกดหน้าสแกนซ้ำ
			\item บรรทัดที่ 25-29 เป็นการหยุดการทำงานของ case เพื่อไม่ให้ทำงานต่อเนื่องและจบการทำงานของเมนูนำทาง 
		\end{itemize}
	
	\section{โครงสร้างของการสร้างหน้า Main2Activity}
	\begin{figure}[H]
		{\setstretch{1.0}\begin{lstlisting}
Button btnSignOut;
FirebaseAuth auth;
FirebaseUser user;
ProgressDialog PD;
	
			\end{lstlisting}}
		\caption{ตัวแปรในคลาส Main2Activity}
		\label{Fig:Main2Activity}
	\end{figure}
	\newpage
	จากภาพที่ \ref{Fig:Main2Activity} ตัวแปรที่ประกาศขึ้นเพื่อใช้ในการทำงานของคลาส FeedFragment สามารถอธิบายได้ดังนี้
	\begin{itemize}[label={--}]
		\item บรรทัดที่ 1 ตัวแปร btnSignOut ใช้ในการแสดงข้อมูลลิสต์รายการข่าวสาร
		\item บรรทัดที่ 2 ตัวแปร auth ใช้จัดเก็บและยืนตัวตนของสมาชิก
		\item บรรทัดที่ 3 ตัวแปร user ใช้จัดการข้อมูลของสมาชิก
		\item บรรทัดที่ 4 ตัวแปร pd ใช้สำหรับเป็นแจ้งเตือน
	\end{itemize}
	\begin{figure}[H]
		{\setstretch{1.0}\begin{lstlisting}
btnSignOut.setOnClickListener(new View.OnClickListener() {
    @Override
    public void onClick(View view) {
        auth.signOut();
        Intent intent = new Intent(Main2Activity.this, LoginActivity.class);
        startActivity(intent);
        finish();
        MainActivity.activityMain.finish();
    }
});

findViewById(R.id.change_password_button).setOnClickListener(new View.OnClickListener() {
    @Override
    public void onClick(View view) {
        startActivity(new Intent(getApplicationContext(), ForgetActivity.class).putExtra("Mode", 1));
    }
});

findViewById(R.id.change_email_button).setOnClickListener(new View.OnClickListener() {		@Override            		public void onClick(View view) {
        startActivity(new Intent(getApplicationContext(), ForgetActivity.class).putExtra("Mode", 2));
    }
});
findViewById(R.id.delete_user_button).setOnClickListener(new View.OnClickListener() {
 @Override            
 public void onClick(View view) {
                startActivity(new Intent(getApplicationContext(), ForgetActivity.class).putExtra("Mode", 3));
            }
        });
    }
			\end{lstlisting}}
		\caption{โค้ดส่วนที่ใช้ในการสร้างเมนูนำทางหลักภายในคลาส Main2Activity}
		\label{Fig:Main2Activity2}
	\end{figure}
	จากภาพที่ \ref{Fig:Main2Activity2} โค้ดส่วนที่ใช้ในการสร้างเมนูนำทางหลักภายในคลาส Main2Activity สามารถอธิบายได้ดังนี้
	\begin{itemize}[label={--}]
		\item บรรทัดที่ 1 ทำหน้าที่ดักจับอีเวนต์ดักจักเมื่อผู้ใช้ทำการกดปุ่ม SIGN OUT
		\item บรรทัดที่ 2-10  เป็นการทำงานเมื่อผู้ใช้กดปุ่ม SIGN OUT  จะทำการออกจากระบบและทำลาย Activity ทั้งหมด
		\item บรรทัดที่ 12-17 เป็นการทำงานเมื่อผู้ใช้กดปุ่ม CHANGE PASSWORD ระบบจะทำการส่งค่า Mode 1
		\item บรรทัดที่ 19-22 เป็นการทำงานเมื่อผู้ใช้กดปุ่ม CHANGE Email ระบบจะทำการส่งค่า Mode 2
		\item บรรทัดที่ 23-29 เป็นการทำงานเมื่อผู้ใช้กดปุ่ม Delete User ระบบจะทำการส่งค่า Mode 3
	\end{itemize}
	
	\section{โครงสร้างของการสร้างหน้า DashboardFragment}
	\begin{figure}[H]
		{\setstretch{1.0}\begin{lstlisting}
List<PieEntry> entries;
PieChart pieChart;
public   static ObjectAnimator progresanimator3, progresanimator2, progresanimator1;
public   static ProgressBar progressBar3, progressBar2, progressBar1;
	\end{lstlisting}}
		\caption{ตัวแปรในคลาส DashboardFragment}
		\label{Fig:DashboardFragment}
	\end{figure}
	จากภาพที่ \ref{Fig:DashboardFragment} ตัวแปรที่ประกาศขึ้นเพื่อใช้ในการทำงานของคลาส DashboardFragment สามารถอธิบายได้ดังนี้
	\begin{itemize}[label={--}]
		\item บรรทัดที่ 1 ตัวแปร entries ใช้สำหรับเก็บข้อมูลลงใน chart
		\item บรรทัดที่ 2 ตัวแปร ObjectAnimator progresanimator3, progresanimator2, progresanimator1 ใช้ตั้งค่าทำให้ chart เคล่ือนไหว
		\item บรรทัดที่ 3 ตัวแปร ProgressBar progressBar3, progressBar2, progressBar1 ใช้ตั้งค่า ProgressBar
	\end{itemize}
	\begin{figure}[H]
		{\setstretch{1.0}\begin{lstlisting}
Date currentTime = Calendar.getInstance().getTime();
SimpleDateFormat df = new SimpleDateFormat("dd-MM-yyyy");
SimpleDateFormat simplemonth = new SimpleDateFormat("MMM-yyyy", Locale.ENGLISH);
SimpleDateFormat simpleyear = new SimpleDateFormat("ww-YYYY");
final String month2 = simplemonth.format(currentTime);
final String year = simpleyear.format(currentTime);
final String date = df.format(currentTime);
	
			\end{lstlisting}}
		\caption{โค้ดส่วนที่ใช้ในการสร้างโหนดวัน สัปดาห์และเดือนภายในคลาส Dashbord}
		\label{Fig:Dashbord}
	\end{figure}
	จากภาพที่ \ref{Fig:Dashbord} โค้ดส่วนที่ใช้ในการสร้างเมนูนำทางหลักภายในคลาส Main2Activity สามารถอธิบายได้ดังนี้
	\begin{itemize}[label={--}]
		\item บรรทัดที่ 1  เป็นการดึงวันที่จากสมาร์ทโฟนมาใช้งาน
		\item บรรทัดที่ 2  เป็นการเลือกรูปแบบวันที่ให้อยู่ในรูปแบบ วัน
		\item บรรทัดที่ 3  เป็นการเลือกรูปแบบวันที่ให้อยู่ในรูปแบบ เดือน
		\item บรรทัดที่ 4  เป็นการเลือกรูปแบบวันที่ให้อยู่ในรูปแบบ สัปดาห์
		\item บรรทัดที่ 5-7 เป็นการแปลงวันที่ให้อยู่ในรูปแบบ String
	\end{itemize}
	\begin{figure}[H]
		{\setstretch{1.0}\begin{lstlisting}
if (status == null) {
 addcal.child("showfooddata").child(user_id).child(date).addListenerForSingleValueEvent(new ValueEventListener() {
	@Override
		public void onDataChange(@NonNull DataSnapshot dataSnapshot) {
		 if (dataSnapshot.getValue() != null) {
			int gscal = Integer.parseInt(dataSnapshot.child("totalcal").getValue().toString());
		if (gsto < 0) {
		    gsto = 0;
	}
		pscal.setText(String.valueOf(gscal));

		chart(gsto, gscal);
		initday2(gsfat, gssod, gssug);
	} else {
		 DatabaseReference zerostatus = addcal.child("showfooddata").child(user_id).child(date);
		 zerostatus.child("totalcal").setValue("0");
		pscal.setText(pcal);
		int gscal = Integer.parseInt(pcal);
						 }
		}
	 });
	 \end{lstlisting}}
		\caption{โค้ดส่วนที่ใช้การตั้งค่า Database}
		\label{Fig:Dashbordset}
	\end{figure}
	\newpage
	จากภาพที่ \ref{Fig:Dashbordset} โโค้ดส่วนที่ใช้การตั้งค่า Database สามารถอธิบายได้ดังนี้
	\begin{itemize}[label={--}]
		\item บรรทัดที่ 1  เป็นการตรวจสอบว่าถ้า Database มีค่าว่าง
		\item บรรทัดที่ 2  เป็นการบันทึกข้อมูลลงใน Database 
		\item บรรทัดที่ 3  เป็นการเรียกใช้เมธอดชันจากคลาสที่ extends มา
		\item บรรทัดที่ 4-11 เป็นการอ่านข้อมุลจาก Database ภายใต้เงื่อนไขว่า ถ้าDatabaseไม่มีค่าว่าง ให้ทำการอ่านข้อมูลและแสดงข้อมูล
		\item บรรทัดที่ 12-13  ใช้สำหรับเรียกใช้เมธอดสร้าง Chart และ ProgressBar 
		\item บรรทัดที่ 14-24 เป็นเงื่อนไขที่ Database มีค่าว่างให้ทำการตั้งค่า Database เป็น 0 แล้วให้ทำการแสดงข้อมูล
	\end{itemize}
		\begin{figure}[H]
		{\setstretch{1.0}\begin{lstlisting}
if (status == null) {
 addcal.child("showfooddata").child(user_id).child(date).addListenerForSingleValueEvent(new ValueEventListener() {
	@Override
		public void onDataChange(@NonNull DataSnapshot dataSnapshot) {
		 if (dataSnapshot.getValue() != null) {
			int gscal = Integer.parseInt(dataSnapshot.child("totalcal").getValue().toString());
		if (gsto < 0) {
		    gsto = 0;
	}
		pscal.setText(String.valueOf(gscal));

		chart(gsto, gscal);
		initday2(gsfat, gssod, gssug);
	} else {
		 DatabaseReference zerostatus = addcal.child("showfooddata").child(user_id).child(date);
		 zerostatus.child("totalcal").setValue("0");
		pscal.setText(pcal);
		int gscal = Integer.parseInt(pcal);
						 }
		}
	 });
	 \end{lstlisting}}
		\caption{โค้ดส่วนที่ใช้ในการแสดงผลหน้าจอ Dashboard}
		\label{Fig:Dashbordset2}
	\end{figure}
	\newpage
	จากภาพที่ \ref{Fig:Dashbordset2} โค้ดส่วนที่ใช้ในการสร้างเมนูนำทางหลักภายในคลาส Main2Activity สามารถอธิบายได้ดังนี้
	\begin{itemize}[label={--}]
		\item บรรทัดที่ 1  เป็นการตรวจสอบว่าถ้า Database มีค่าว่าง
		\item บรรทัดที่ 2  เป็นการบันทึกข้อมูลลงใน Database 
		\item บรรทัดที่ 3  เป็นการเรียกใช้เมธอดชันจากคลาสที่ extends มา
		\item บรรทัดที่ 4-11 เป็นการอ่านข้อมุลจาก Database ภายใต้เงื่อนไขว่า ถ้าDatabaseไม่มีค่าว่าง ให้ทำการอ่านข้อมูลและแสดงข้อมูล
		\item บรรทัดที่ 12-13  ใช้สำหรับเรียกใช้เมธอดสร้าง Chart และ ProgressBar 
		\item บรรทัดที่ 14-24 เป็นเงื่อนไขที่ Database มีค่าว่างให้ทำการตั้งค่า Database เป็น 0 แล้วให้ทำการแสดงข้อมูล
	\end{itemize}



	\section{โครงสร้างของการสร้างหน้า ScanFragment}
	\begin{figure}[H]
		{\setstretch{1.0}\begin{lstlisting}
public CameraKitView cameraKitView;
public static final String MODEL_PATH = "optimized_graph.lite";
public static final String LABEL_PATH = "retrained_labels.txt";
private Classifier classifier;
			\end{lstlisting}}
		\caption{ตัวแปรในคลาส ScanFragment}
		\label{Fig:ScanFragment}
	\end{figure}
	จากภาพที่ \ref{Fig:ScanFragment} ตัวแปรที่ประกาศขึ้นเพื่อใช้ในการทำงานของคลาส ScanFragment สามารถอธิบายได้ดังนี้
	\begin{itemize}[label={--}]
		\item บรรทัดที่ 1 ตัวแปร cameraKitView; ใช้สำหรับสร้างกล้องเพื่อทำการถ่ายภาพ
		\item บรรทัดที่ 2 ตัวแปร String MODEL\_PATH = optimized\_graph.lite;  ใช้สำหรับระบุที่อยู่และชื่อของไฟล์ lite
		\item บรรทัดที่ 3 ตัวแปร String LABEL\_PATH = retrained\_labels.txt;  ใช้สำหรับระบุที่อยู่และชื่อของไฟล์ txt
		\item บรรทัดที่ 4 ตัวแปร classifier ใช้สำหรับเรียกใช้คลาส classifier
	\end{itemize}

	\begin{figure}[H]
		{\setstretch{1.0}\begin{lstlisting}
cameraKitView.captureImage(new CameraKitView.ImageCallback() {
	@Override
	public void onImage(CameraKitView cameraKitView, final byte[] photo) {
	Bitmap bitmap = BitmapFactory.decodeByteArray(photo, 0, photo.length);
	bitmap = Bitmap.createScaledBitmap(bitmap, INPUT_SIZE, INPUT_SIZE, false);
	final List<Classifier.Recognition> results = classifier.recognizeImage(bitmap);
	tvLabel.setText(results.get(0).getTitle());
	final String ss = tvLabel.getText().toString().trim();
	DatabaseReference dbfood = FirebaseDatabase.getInstance().getReference();
dbfood.child("foods").orderByKey().equalTo(ss).addListenerForSingleValueEvent(new ValueEventListener() {
@Override
	public void onDataChange(@NonNull DataSnapshot dataSnaps
	Intent intent = new Intent(getActivity(), Food.class);
	for (DataSnapshot ds : dataSnapshot.getChildren()) {
		String ds1 = ds.getKey();
	intent.putExtra("Name", dataSnapshot.child(ds1).child("Name").getValue().toString());
	}
getContext().startActivity(intent)
}
\end{lstlisting}}
		\caption{โค้ดส่วนที่ใช้ในการสแกน}
		\label{Fig:ScanFragment2}
	\end{figure}
	\newpage
	จากภาพที่ \ref{Fig:ScanFragment2} โค้ดส่วนที่ใช้ในการสแกนสามารถอธิบายได้ดังนี้
	\begin{itemize}[label={--}]
		\item บรรทัดที่ 1  เป็นการใช้ฟังก์ชันถ่ายรูป
		\item บรรทัดที่ 2  เป็นการเรียกใช้เมธอดชันจากคลาสที่ extends มา 
		\item บรรทัดที่ 3-6 เป็นการทำการ Recognation เพื่อนำภาพที่ถ่ายไปจำแนก
		\item บรรทัดที่ 7  เป็นการนำค่าที่ได้จากการจำแนกมาแสดง
		\item บรรทัดที่ 9-14  เป็นการอ่านข้อมูลไปยัง Database โดยการนำค่าที่ได้จากการจำแนกไปเทียบกับค่า Key ใน Database 
		\item บรรทัดที่ 15-19 เมื่อค่าจากากการจำแนกตรงกับค่า Key ให้ทำการส่งข้อมูลไปยัง คลาส Food แล้วทำการเปิดคลาส Food
	\end{itemize}





	\section{โครงสร้างของการสร้างหน้า AddfoodFragment}
	\begin{figure}[H]
		{\setstretch{1.0}\begin{lstlisting}
ArrayList<Foodmanul> foodlist;
FirebaseDatabase addfood = FirebaseDatabase.getInstance();
			 
			\end{lstlisting}}
		\caption{ตัวแปรในคลาส AddfoodFragment}
		\label{Fig:Addfood1}
	\end{figure}
	จากภาพที่ \ref{Fig:Addfood1} ตัวแปรที่ประกาศขึ้นเพื่อใช้ในการทำงานของคลาส AddfoodFragment สามารถอธิบายได้ดังนี้
	\begin{itemize}[label={--}]
		\item บรรทัดที่ 1 ตัวแปร cameraKitView; ใช้สำหรับสร้างกล้องเพื่อทำการถ่ายภาพ
		\item บรรทัดที่ 2 ตัวแปร foodlist; จัดเก็บข้อมูลอาหาร
	
	\end{itemize}

	\begin{figure}[H]
		{\setstretch{1.0}\begin{lstlisting}
addcal.addValueEventListener(new ValueEventListener() {
@Override
public void onDataChange(@NonNull DataSnapshot dataSnapshot) {
	foodlist.clear();
		int i = 0;
		for (DataSnapshot foodSnapshot : dataSnapshot.getChildren()) {
				foodlist.add(foodSnapshot.getValue(Foodmanul.class));
				foodlist.get(i).setKey(foodSnapshot.getKey());
				i++;
		}
	listViewfood.setLayoutManager(new LinearLayoutManager(getContext()));
	listViewfood.setHasFixedSize(true);
	FoodList adapter = new FoodList(getContext(),foodlist,putex);
	listViewfood.setAdapter(adapter);
}

\end{lstlisting}}
		\caption{โค้ดส่วนที่ใช้แสดงผลหน้าจอเพิ่มอาหาร}
		\label{Fig:Addfood3}
	\end{figure}
	\newpage
	จากภาพที่ \ref{Fig:Addfood3} โค้ดส่วนที่ใช้แสดงผลหน้าจอเพิ่มอาหาร สามารถอธิบายได้ดังนี้
	\begin{itemize}[label={--}]
		\item บรรทัดที่ 1  เป็นเมธอดการอ่านค่าจาก Database
		\item บรรทัดที่ 2  เป็นการเรียกใช้เมธอดชันจากคลาสที่ extends มา 
		\item บรรทัดที่ 3-10 อ่านข้อมูลจากคลาส Foodmanul
		\item บรรทัดที่ 11-14 เป็นการนำข้อมูลที่ได้มาแปลงเป็นลิสต์และแสดงผล

	\end{itemize}




	\section{โครงสร้างของคลาส FoodList}
	\begin{figure}[H]
		{\setstretch{1.0}\begin{lstlisting}
CardView cardViewfood;
ArrayList<Foodmanul> foodlists;
String user_id;

			\end{lstlisting}}
		\caption{ตัวแปรในคลาส FoodList}
		\label{Fig:foodlist1}
	\end{figure}
	จากภาพที่ \ref{Fig:foodlist1} ตัวแปรที่ประกาศขึ้นเพื่อใช้ในการทำงานของคลาส FoodList สามารถอธิบายได้ดังนี้
	\begin{itemize}[label={--}]
		\item บรรทัดที่ 1 ตัวแปร cardViewfood;เป็นตัวแสดง​ CardView
		\item บรรทัดที่ 2 ตัวแปร foodlist; ใช้จัดเก็บข้อมูลอาหารจากคลาส Foodmanul
		\item บรรทัดที่ 3 ตัวแปร user\_id; ใช้เก็บค่า ID ของสมาชิก
	\end{itemize}

	\begin{figure}[H]
		{\setstretch{1.0}\begin{lstlisting}
Buutonupdaet.setOnClickListener(new View.OnClickListener() {
	@Override
	public void onClick(View v) {
		String name = editTextName.getText().toString().trim();
			String cal = editTextCal.getText().toString().trim();
	if(TextUtils.isEmpty(name)){
		editTextName.setError("enter name pls");
	return;
   }
 		updatefood(name,cal,Sug,Sod,Fat,key);
				alertDialog.dismiss();
	}
		});
	Buttondelete.setOnClickListener(new View.OnClickListener() {
	@Override
	public void onClick(View v) {
		deletefood(key);
			alertDialog.dismiss();
			}
		});
\end{lstlisting}}
		\caption{โค้ดการทำงานของคลาส Foodlist}
		\label{Fig:foodlis1}
	\end{figure}
	\newpage
	จากภาพที่ \ref{Fig:foodlis1} โค้ดการทำงานของคลาส Foodlistสามารถอธิบายได้ดังนี้
	\begin{itemize}[label={--}]
		\item บรรทัดที่ 1-5 เป็นการดักจับอีเวนต์เมื่อผู้ใช้ทำการกดปุ่ม update
		\item บรรทัดที่ 6-9  เป็นการดักจับค่าว่างในกรณีที่สมาชิกไม่ทำการใส่ข้อความแล้วให้แสดงแจ้งเตือนกลับมา
		\item บรรทัดที่ 10-13 เรียกใช้เมธอด updatefood แล้วทำการปิด Dialog
		\item บรรทัดที่ 14-17  เป็นการดักจับอีเวนต์เมื่อผู้ใช้ทำการกดปุ่ม delete โดยเลือกลบที่ Key
		\item บรรทัดที่ 18-20  ทำการปิด Dialog
\end{itemize}


\section{โครงสร้างของคลาส LoginActivity}
\begin{figure}[H]
	{\setstretch{1.0}\begin{lstlisting}
private FirebaseAuth auth;
private ProgressDialog PD;
		\end{lstlisting}}
	\caption{ตัวแปรในคลาส LoginActivity}
	\label{Fig:LoginActivity}
\end{figure}
จากภาพที่ \ref{Fig:LoginActivity} ตัวแปรที่ประกาศขึ้นเพื่อใช้ในการทำงานของคลาส LoginActivity สามารถอธิบายได้ดังนี้
\begin{itemize}[label={--}]
	\item บรรทัดที่ 1 ตัวแปร auth; ใช้จัดเก็บและยืนตัวตนของสมาชิก
	\item บรรทัดที่ 2 ตัวแปร PD;  ใช้แสดงข้อความให้กับสมาชิก 
	
\end{itemize}

\begin{figure}[H]
	{\setstretch{1.0}\begin{lstlisting}
auth.signInWithEmailAndPassword(email, password)
	.addOnCompleteListener(LoginActivity.this, new OnCompleteListener<AuthResult>() {
	@Override
			public void onComplete(@NonNull Task<AuthResult> task) {
	if (task.isSuccessful()) {
		Intent intent = new Intent(LoginActivity.this, MainActivity.class);
	startActivity(intent);
		finish();
	} else {
	Toast.makeText(
	LoginActivity.this,
	"Authentication Failed",
Toast.LENGTH_LONG).show();
Log.d("error", task.getException().toString());
}
	PD.dismiss();
			}
	});
\end{lstlisting}}
	\caption{โค้ดส่วนที่ใช้ในการเข้าสู่ระบบในคลาส LoginActivity}
	\label{Fig:LoginActivity2}
\end{figure}
จากภาพที่ \ref{Fig:LoginActivity2} โค้ดส่วนที่ใช้ในการเข้าสู่ระบบในคลาส LoginActivity สามารถอธิบายได้ดังนี้
\begin{itemize}[label={--}]
	\item บรรทัดที่ 1-3 เป็นการตรวจสอบ Email และ Password ของผู้ใช้
	\item บรรทัดที่ 4-8 เป็นการทำงานมนกรณีที่เข้าสู่ระบบสำเร็จจะทำการเรียกใช้คลาส MainActivity
	\item บรรทัดที่ 9-18 เป็นการทำงานเมื่อเข้าสู่ระบบไม่สำเร็จจะทำการแสดงข้อความว่า Authentication Failed
	
\end{itemize}