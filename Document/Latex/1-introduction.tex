\chapter{บทนำ}

\section{ที่มาและเหตุผล }
ทุกวันนี้ประเทศไทยมีอาหารที่หลากหลาย แต่อาหารบางชนิดยังมีการปรุงอาหารรสจัดมากจนเกินไป ทำให้ร่างกายได้รับสารอาหารมากเกินความจำเป็น ส่งผลให้ร่างกายเกิดโรคตามมา เช่น การปรุงอาหารรสเค็มจัด ก็จะทำให้ผู้บริโภคมีความเสี่ยงต่อการเป็นโรคไต จึงมีการพัฒนาแอปพลิเคชันมาใช้ในการจัดเก็บข้อมูลการบริโภคทำให้สามารถควบคุมไม่ให้มีการบริโภคเกินความจำเป็นของร่างกาย แต่แอปพลิเคชันยังมีข้อเสียคือ แอปพลิเคชันจะต้องมีการซื้อเพื่อที่จะสามารถใช้งานได้เต็มประสิทธิภาพและยังมีความซับซ้อนในการบันทึกข้อมูลการบริโภค

แต่ในปัจจุบันมีการนำปัญญาประดิษฐ์(Artificial Intelligence:AI) มาใช้ประโยชน์อย่างแพร่หลาย ไม่ว่าจะเป็นทางด้านการแพทย์เพื่อช่วยในการวินิจฉัยโรค ด้านการคมนาคมช่วยมาการมาช่วยในการควบคุมการทำงานของยานพาหะนะหรือด้านการเกษตรที่ช่วยในการควบคุมระบบน้ำและอุณหภูมิ แต่ทางด้านอาหารยังไม่มีการนำมาใช้มากนัก 

ผู้พัฒนาจึงเล็งเห็นปัญหาดังกล่าวจึงได้ทำการพัฒนาแอปพลิเคชันจัดเก็บข้อมูลการบริโภคงานวิจัยนี้ได้มีการนำปัญญาประดิษฐ์มาช่วยในการบันทึกข้อมูลการบริโภคโดยระบบจำทำการเรียนรู้ภาพที่ทำการถ่ายเข้ามาแล้วนำมาประมวลผลเพื่อระบุข้อมูลอาหารจากภาพถ่าย งานวิจัยนี้ได้นำเทคนิคทางด้านการประมวลผลภาพ(Image Processing) และเทคนิคการเรียนรู้ของเครื่อง (Machine Learning) มาใช้ในการตรวจจับอาหาร



\section{วัตถุประสงค์}
\begin{enumerate}
	\item เพื่อพัฒนาแอพพลิเคชั่นสแกนอาหารสำหรับเก็บและแสดงข้อมูลการบริโภคอาหาร 
	\item  เพื่อสามารถบอกชื่ออาหารที่ทำการสแกนได้อย่างแม่นยำ  
\end{enumerate}
% \section{ขอบเขตของโครงงาน}
% \begin{enumerate}[label=1.3.\arabic*]
% 	\begin{enumerate}
% 		\item ใช้ภาพถ่ายอาหาร ที่ขายดีในเซเว่น อีเลฟเว่น จำนวน 50 ชนิด เพื่อใช้ในการฝึกสอน(train)
% 		\item อาหารแต่ละชนิดจะใช้ภาพถ่ายจำนวน 100 รูป

% 	\end{enumerate}
% \end{enumerate}


\newpage

\section{ขอบเขตของโครงงาน}
\begin{enumerate}
	\item ใช้ภาพถ่ายอาหาร ที่ขายดีในเซเว่น อีเลฟเว่น จำนวน 50 ชนิด เพื่อใช้ในการฝึกสอน(train)
		\item อาหารแต่ละชนิดจะใช้ภาพถ่ายจำนวน 100 รูป 
\end{enumerate}
\section{ประโยชน์ที่คาดว่าจะได้รับ}
\begin{enumerate}
	\item ผู้ใช้ได้รับความสะดวกในการเก็บข้อมูลการบริโภคของตนเอง 
	\item ผู้ใช้มีการบริโภคอาหารที่ดีขึ้น
  \item เป็นการวิจัยที่สามารถนำไปต่อยอดได้โดยการเพิ่มฟังก์ชันการออกกำลังการเพื่อให้มีการดูแลสุขภาพทั้งทางด้านการกินและการออกกำลังกาย 
\end{enumerate}
\section{เครื่องมือที่ใช้ในการพัฒนา (Development tools)}
\subsection{ฮาร์ดเเวร์}
\begin{enumerate}
	\item สมาร์ทโฟน (Smart phone)
		\begin{itemize}
			\item ระบบปฏิบัติการเวอร์ชัน 9.0 หรือ API Level 21
			\item หน่วยประมาลผลกลาง Qualcomm Snapdragon 660 AIE Octa Core ความเร็ว 2.2 กิกะเฮิร์ตซ์ (Gigahertz, GHz)
			\item หน่วยความจำหลักอย่างน้อย 4 กิกะไบต์ (Gigabyte, GB)
			\item หน่วยความจำสำรองอย่างน้อย 32 กิกะไบต์ (Gigabyte, GB)
			\item หน้าจอแสดงผลความละเอียดอย่างน้อย 2160 x 1080 พิกเซล  (Pixel)
			\item หน้าจอแสดงผลขนาดอย่างน้อย 5.99 นิ้ว
			\item กล้องถ่ายรูปความละเอียดอย่างน้อย 20 เมกกะพิกเซล (Magapixel)
		\end{itemize}
	
	\item เครื่องคอมพิวเตอร์ส่วนบุคคล (Personal computer)
		\begin{itemize}
			\item  	ระบบปฏิบัติการ macOS mojave 
			\item  หน่วยประมวลผลกลาง(cpu) Intel Core i5  ความเร็ว 2.3 กิกะเฮิร์ตซ์ (Gigahertz, GHz)
			\item  หน่วยประมวลผลกราฟฟิก Intel Iris Plus Graphics 640 ความจำ 1.5 กิกะไบต์ (Gigabyte, GB) 
			\item  หน่วยความจำหลัก 8 กิกะไบต์ (Gigabyte, GB)
			\item  หน่วยความจำสำรอง 256 กิกะไบต์ (Gigabyte, GB)
		\end{itemize}
\end{enumerate}

\subsection{ซอฟต์แวร์ (Software)}
\begin{enumerate}
	\item Java เป็นภาษาโปรแกรมที่ใช้ในการพัฒนา 
	\item Android Studio เป็น IDE (Integrated Development Environment) ใช้พัฒนาแอปพลิเคชันสำหรับระบบปฏิบัติการแอนดรอยด์
	\item Tensorflow Lite เป็น Libary ใช้ในการสร้างโมเดล ในการทำ Machine Learning 
\end{enumerate}

\newpage
\subsection{แผนการดำเนินการ}
	ในการพัฒนาแอปพลิเคชันสแกนอาหาร ผู้พัฒนาได้แบ่งขั้นตอนการดำเนินงานไว้ด้วยกัน 8 ขั้นตอน ดังต่อไปนี้

%\begin{landscape}
%\sffamily
\begin{table}[H]
	\noindent
	\caption{ขั้นตอนการดำเนินงาน}
	\begin{ganttchart}[
		canvas/.append style={fill=none, draw=black!5, line width=.75pt},
		vgrid={*2{draw=black!7, line width=.75pt}},
		title label font=\bfseries\footnotesize,
		bar label node/.append style={
			align=left,
			text width=width("7. Functional Testing On")},
		bar/.append style={draw=none, fill=black!63}
		]{1}{18}
		\gantttitle{2561}{6}
		\gantttitle{2562}{12}\\
		\gantttitle{พ.ย.}{3}
		\gantttitle{ธ.ค.}{3}
		\gantttitle{ม.ค.}{3}
		\gantttitle{ก.พ.}{3} 
		\gantttitle{มี.ค.}{3}
		\gantttitle{เม.ย.}{3} \\
		\ganttbar{1.ศึกษาความเป็นไปได้}{1}{6} \\
		\ganttbar{2.เสนอหัวข้อโครงงาน}{1}{3} \\
		\ganttbar{3.ศึกษาค้นคว้าข้อมูล}{3}{6} \\
		\ganttbar{4.ศึกษาการใช้เครื่องมือ}{1}{6} \\
		\ganttbar{5.วิเคราะห์และออกแบบ}{6}{9} \\
		\ganttbar{6.เขียนโปรแกรม}{6}{15} \\
		\ganttbar{7.ทดสอบและแก้ปัญหา}{10}{16} \\
		\ganttbar{8.จัดทำเอกสาร}{17}{18} \\
	\end{ganttchart}
	\label{tab:ganttchart}
\end{table}
%\end{landscape}
%TODO แก้เทมเพลตเอาชื่อตารางไว้ด้านบน
